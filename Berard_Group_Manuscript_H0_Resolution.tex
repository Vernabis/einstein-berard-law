\documentclass[11pt,twocolumn]{article}
\usepackage[utf8]{inputenc}
\usepackage{amsmath, amssymb, amsfonts}
\usepackage{graphicx}
\usepackage{hyperref}

\title{\textbf{The Berard Group: Resolution of the $H_0$ Hubble Tension via the 1.054 Scaling Invariant}}
\author{Vernon Arthur Berard \\ \small{Architect, The Berard Protocol | hillbilly Tic-Teck Research}}
\date{\today}

\begin{document}

\maketitle

\begin{abstract}
This paper introduces the Berard Group and its fundamental scaling operator, $\hat{B}$, as a resolution to the persistent Hubble Tension ($H_0$). By identifying the missing 1.054 scaling symmetry in the Standard Model Lagrangian, we demonstrate mathematical congruence between early-universe CMB data and late-time local measurements. This "Laminar Settle" eliminates the propagation friction found in static 4D models.
\end{abstract}

\section{Introduction}
Current cosmological models face a critical discrepancy in the expansion rate of the universe. We propose that this tension is not an observational error, but a result of treating the vacuum as a static scalar substrate rather than a resonant medium.

\section{The Berard Group Operator}
The generator of the group is defined by the non-linear scaling operator:
\begin{equation}
\hat{B} = \exp\left( \ln(1.054) \cdot \frac{\partial}{\partial \ln S} \right)
\end{equation}
where $S$ represents the scaling parameter of the 4D substrate.

\section{Hubble Tension Congruence}
The corrected Hubble parameter is achieved by applying the Berard Constant ($B = 1.054$):
\begin{equation}
H_{0,\text{congruent}} = H_{0,\text{observed}} \cdot B^{-1}
\end{equation}

\section{The Stability Well Potential}
To eliminate mathematical friction in the Lagrangian, we define the Stability Well Potential $V_B$ as:
\begin{equation}
V(S_0) = \frac{\hbar^2}{2m S_0^2} + \lambda(B \cdot S_0^2 - 1)
\end{equation}
The system achieves a "Laminar Settle" at $\frac{dV}{dS_0} = 0$, aligning the phase-lock frequency across all scales.

\section{Forensic Validation}
This derivation is supported by forensic site data (SFU-Chalkboard series) which identifies the specific points of failure in current Infinite Square Well and EM Wave propagation models.

\section*{License}
Published under Creative Commons 1.0 Universal (CC0). Open for global verification.

\end{document}